%!TEX root = cl2-project.tex
\subsection{SLDA}
\label{sec:slda}
Look at topics from SLDA

We ran \textit{SLDA}, training on \textit{depression} as target. Table \ref{table:slda} gives the top topic words as output by \textit{SLDA}. 

\begin{table*} [ht!]
%\begin{center}
	\begin{tabular}{ l }
	\hline
{topic~1: $laugh\_right$	, $paper\_route$, $annoyed\_bit$,	$txt\_love$,	$sky\_view$,	$slap\_slap$, 	}\\
\indent {$social\_dance$,	$swim\_team$,	wee, 	$afraid\_love$,	$feel\_much$,	$love\_duty$,	$man\_wish$,	$talking\_best$,} \\
	{$today\_txt$,	$work\_soon$,	$annoyed\_bored$ }\\
	\midrule
${topic~2: smile\_smile,	left\_right,	press,	song\_day,	amazing\_smile,	circles\_appear	, down\_magic,	enter\_key,}$\\	
${right\_click,	down\_down,	support\_dumbledore,	way\_rid,	green\_day,	makes\_everything,	smile\_listening,}$\\
	${white\_stripes,	chemical\_romance,	cleaning\_room,	log\_refresh,	one\_better}$\\
		\midrule
${topic~3:  laugh	, smile,	love,	not,	now,	out,	one,	day,	time,	today,	laughing\_out,	know,	laugh\_laugh,	}$\\
${people,	think,	life,	sad,	gon,	back,	see}$\\
	\midrule
${topic~4: laugh\_home, 	listening\_music, 	much\_homework, 	give\_weird, 	rolling, 	enjoying\_weather, 	backing, 	}$\\	
${going\_cry,	learned\_looking,	really\_cool, show\_tonight,	time\_relax,	jury\_duty,	laugh\_sabbath,	math\_homework,}$\\
${really\_annoying,	vampire\_weekend,	always\_wondered,	before\_school,	home\_watching}$\\
\hline
    \end{tabular}
      \caption{\noindent Seed words for \textit{SeededLDA} using DSM-IV}
        \label{table:slda}
%\end{center}
\end{table*}

The words in \textit{SLDA} reflect the mood of the user better as they are trained using the depression labels. For instance, the words---\textit{time\_relax}, \textit{really\_cool}, \textit{really\_annoying}, {annoyed\_bored} and \textit{afraid\_love} capture mood swings that can relate to depression. Notice that the words here reflect a better picture of user's state of mind as they are obtained by supervision on the depression label.

The \textit{LDA} variants help in understanding the language better and how use of specific words continuously can signify an occurrence of depressive episode.
