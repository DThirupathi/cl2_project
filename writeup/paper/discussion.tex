%!TEX root = cl2-project.tex
\section{Discussion}
\label{sec:disc}

In this work, we take steps toward understanding how the language analysis can uncover information about people's mental state. We explored \textit{topic modeling} and \textit{SeededLDA} as means to model user status messages which stem from their state of mind. This has helped us understand the different feelings expressed by users online. \textit{SeededLDA} captures the polarity of emotions which was helpful in understanding the mental state of the user. Seeding topic models with words from the  DSM-IV manual was very helpful in encoding the symptoms that medical practitioners use to detect depression is a easy manner. Some of these include changes in sleep and eating habits, which people are likely to complain about in Facebook statuses. a more careful selection of seeds can hence unravel more information about the individual and narrow down symptoms that can lead to a depressive episode. A more detailed analysis of language can hence be helpful to detect depressive symptoms before a major depressive episode strikes.