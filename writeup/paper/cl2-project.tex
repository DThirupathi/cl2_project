%
% File acl2014.tex
%
% Contact: koller@ling.uni-potsdam.de, yusuke@nii.ac.jp
%%
%% Based on the style files for ACL-2013, which were, in turn,
%% Based on the style files for ACL-2012, which were, in turn,
%% based on the style files for ACL-2011, which were, in turn, 
%% based on the style files for ACL-2010, which were, in turn, 
%% based on the style files for ACL-IJCNLP-2009, which were, in turn,
%% based on the style files for EACL-2009 and IJCNLP-2008...

%% Based on the style files for EACL 2006 by 
%%e.agirre@ehu.es or Sergi.Balari@uab.es
%% and that of ACL 08 by Joakim Nivre and Noah Smith

\documentclass[11pt]{article}
\usepackage{nips13submit_e,times}
\usepackage{times}
\usepackage{url}
\usepackage{latexsym}
\usepackage{amsmath}
\usepackage{colortbl}
\usepackage{graphicx}
\usepackage{booktabs}
\usepackage{caption}
\usepackage{subfigure}
\usepackage{balance}
\usepackage{tabularx}
\usepackage{multirow}
\usepackage{lipsum}

\newcommand{\blue}[1]{{\color{blue} #1}}
%\setlength\titlebox{5cm}

% You can expand the titlebox if you need extra space
% to show all the authors. Please do not make the titlebox
% smaller than 5cm (the original size); we will check this
% in the camera-ready version and ask you to change it back.


\title{Predicting Depression in Facebook Users using Language Analysis}

\date{}

\begin{document}
\maketitle
\begin{abstract}
In this work, we explore machine learning models which automatically learn representations of the data in an attempt to understand depression symptoms among Facebook users. Topic models, and recent models proposed in the field of deep neural networks, learn a representation of the data which has been shown to contain rich semantic information.
%In this work, we explore variations of topic-modeling in an attempt to understand depression symptoms and use that to predict depression. 
We propose a model for predicting depression among Facebook users using their status updates %drawing features from user behavior, topic modeling and distributed word representations learnt from neural networks, 
and demonstrate how using these machine learning models help in gaining a deeper understanding of user behavior leading to better prediction results.
\end{abstract}

%!TEX root = cl2-project.tex
\section{Introduction}
\label{sec:introduction}

Identifying depression symptoms is a challenging problem faced by health practitioners. About 25 million adults suffer from symptoms of depression in the United States \cite{NAMI2013}. A lot of these go undetected because often people don't seek medical help when confronted with these symptoms. Symptoms that suggest depression include insomnia, urge to cry, loss of appetite, weight loss/gain and since these can occur due to changes in routine, work demands, depression is often mistaken and causes it to go unreported. Also, so far, the clinical assessments for depression (e.g. The Minnesota Multiphasic Inventory, MMPI) have been based on patient's self accounts of depression associated symptoms and this relies heavily on patients' ability to recognize and report symptoms. Suppression of these symptoms is also an issue among individuals in positions that demand lack of depression (e.g. pilots, army officials). 

Alternative methods for evaluating depression are on the rise, which are more far-reaching than the traditional methods. With the outbreak of the social network, there is data from social networks which contain a trace of online activities. With careful examination of this data, people under higher risk of depression can be identified and counseled before they are stricken by a massive depression episode. Some of these methods include the strategic placement of online apps in sites frequented by large number of people and collecting data about changes in their behavior without too much intrusion. One such app is the Facebook app. This app presents a few questions regarding their daily routine, sleeping and eating habits and then uses that information to predict if the person has a risk of depression. 


Though the app is not that intrusive, it only can predict depression for users that use the app truthfully. Hence, predicting risk of depression using user statuses, change in the choice of words and tone is a more non-intrusive way of predicting depression and is even more far-reaching than the app. To this end, in this work, we examine Facebook statuses of users and employ computational linguistics methods for predicting risk of depression. We use supervised machine learning methods to find language features that correlate heavily with depression risk as predicted by the app to learn language indicators that are suggestive of depression.
We explore variations of topic modeling---\textit{seeded topic models} \cite{jagarlamudi12} for understanding the text. We demonstrate that these linguistic tools help capture depression symptoms and help in predicting risk of depression.

The rest of the paper is organized as follows. We start by reviewing some of the existing work in Section \ref{relatedwork}. In Section \ref{sec:problem} we explain the problem more formally, elaborate on the prediction task. In Section \ref{sec:baseline}, we explain the surface level features in our model. We discuss the dataset and preprocessing steps in \ref{sec:data}. In Sections \ref{sec:lda} and \ref{sec:seededlda}, we explore some variants of \textit{topic modeling}. This is followed by an experimental section \ref{sec:results}, where we present quantitative results and analysis of features used in our models. We finally wrap up by discussing future work and next steps in Section \ref{sec:discussion}.


%!TEX root =cl2-project.tex
\section{Related Work}
\label{sec:relatedwork}

In this section, we review some of the existing work on using computational linguistics methods for analyzing emotional disorders. \cite{Resnik et. al.} used LIWC and LDA features to predict neuroticism in college students.  
%!TEX root = cl2-project.tex
\section{Problem Statement}
\label{sec:problem}


In this work, we are working with data collected from a Facebook app called \textit{MyPersonality}. The \textit{MyPersonality} project is an app which presents users invoking the app with a questionnaire for assessing personality, emotional stability, depression. When people participated in the questionnaire, they also allowed the app to access their Facebook statuses. In this work, we concentrate on the depression questionnaire. The data hence, consists of Facebook statuses and a score upon completion of the depression questionnaire. The score sums up all the responses to the questions and a threshold on the final score is chosen to label the participants with a risk of depression. 
We treat this as a supervised classification task, using the label given by the \textit{MyPersonality} dataset. We construct two types of features --- 1) surface level features such as gender, unigram and bigram features from Facebook statuses, time of day the post was made,  and 2) language features, for which we explore some variants of \textit{LDA}. Using the features we construct models for predicting depression.
%!TEX root = cl2-project.tex
\section{Dataset}
\label{sec:dataset}
As a part of training data we were given 127216 Facebook statuses belonging to 700 users. For each status we were also given the time at which the status was posted. For each user we had the following metadata:
\begin{enumerate}
\item Gender
\item CES-D score
\item Individual scores for the 20 questions
\end{enumerate}
Apart from this, we were also given  X Facebook statuses belonging to 12540 users without their labels (we call this the unlabeled training data)\\\\
As a part of testing data, we were given the same metadata for 239 users and asked to predict their labels.
%!TEX root = cl2-project.tex
\section{Pre-processing}
\label{sec:preprocessing}

The data that was provided to us were Facebook status messages of users. Status messages posted in Facebook are inherently noisy and thus deserves careful processing. We list the various preprocessing techniques that we applied on the status data to get rid of the noise from it. 
\begin{itemize}
\item[(i)] {\bf Emoticons}: We replaced the emoticons using an emoticon dictionary that we constructed from~\cite{emoticondict}. For example, whenever we encountered any of these symbols \texttt{:-) :) :o) :] :3 :c) :> =] 8) =) (:}, we replaced it with the word {\bf smile}.
\item[(ii)] {\bf Stopwords, Punctuations \& Numbers}: Stopwords, punctuations and alphanumeric characters from the status messages were removed as they do not carry much information on their own.
\item[(iii)] {\bf Acronyms}: Short messages esp. the staus messages posted in Facebook tend to contain lots of acronyms like LOL meaning laugh out loud, ROFL meaning rolling on the floor laughing, etc. We construct an acronym dictionary and use it for replacing acronyms with their expanded versions.
\item[(iv)] {\bf Expressions}: Expressions such as ``hahahahhahahahaha'', ``hehehehhehhehehhehehe'', ``yayyyyyyyyyyyy'' are very common across status messages in Facebook. The best way to deal with these words or expressions is to user regular expressions for detection of such patterns and replace them with words like {\bf laugh}, {\bf happy}, etc. Further words containing repeated characters (with 3 occurences or more) like ``happpppyyyyyyyy'' were reduced to {\bf happyy} as it is not common or perhaps impossible to have three consecutive occurences of the same letter in a word in English language. Therefore any such occurence were reduced to a sequence of two consecutive letters and then our spell-checker (described next) was applied on it to get the correct word.
\item[(v)] {\bf Spelling correction}: For words not found in the vocabulary after applying the previous preprocessing steps, we correct them using a spell-checker that was built using ideas from~\cite{spellcheck}. Essentially, in the spell-checker we are trying to find the correction $c$, out of all possible corrections, that maximizes the probability of $c$ given the original incorrect word $w$:
\begin{equation*}
 \mbox{argmax}_c P(c|w)
\end{equation*}
which is equivalent to 
\begin{equation*}
\mbox{argmax}_c P(w|c) P(c)
\end{equation*}
Here $P(c)$ is the language model, which is the likelihood of occurrence of $c$ in an English text. We use an unigram model to calculate the probability of $c$ from the Brown corpus. To provide better estimates for $P(c)$, we could have used a bigram model here, however in that case we would probably require a ``status message'' corpus. Next, we need to figure out $P(w|c)$, the probability that $w$ would be typed in a text when the user meant $c$, which is the error model. Finally, we need to enumerate all feasible values of $c$ and then choose the one that gives the best combined probability score. 


\end{itemize}















%!TEX root = cl2-project.tex
\section{Baseline Model}
\label{sec:baseline}

\subsection{Baseline}
The  baseline system had the following features:
\begin{enumerate}
  \item Unigram and Bigram: Bag of words approach has been known to provide a significant feature set for many of the language modeling task. But we wanted to include only those words that were highly indicative of the label that we are trying to predict. For this, we first ranked the unigrams according to their frequency and took the top 6000 unigrams. Out of these, we selected those which correlated more than 0.1 as per Pearson's rank correlation with the label we are tring to predict. This gave us a total of 112 unigrams. We did a similar filtering for the bigrams and got a set of 38 bigrams. These 112 unigrams and 38 bigrams corresponded to a set of 150 features where each feature value was the count of that unigram (or bigram) present in the concatenated statuses of that user.
  \item Gender: Kenneth et.al [ref] studied that one of the potential risk factors for major depression is female sex. We included the gender as a binary feature.
  \item NRC word-emotion association lexicon: The lexicon is a list of English words and their associations with eight emotions (anger, fear, anticipation, trust, surprise, sadness, joy, and disgust). We included eight features corresponding to the eight emotions and the feature value for a particular emotion was the sum of the association measures of all the words in concatenated status with that emotion.
  \item Time window:  We included 5 features  related to the time of the status update (1) frequency of status updates per day, (2) number of statuses posted between 6-11 am, (3) number of statuses posted between 11-16, (4) number of sta- tuses posted between 16-21, (5) number of statuses posted between 21-00, and (6) number of statuses posted be- tween 00-6 am.
  \item Topic modeling: We used vanilla LDA as implemented by the mallet toolkit. We concatenated all the statuses of a single user into one document. We need this for both the users in training dataset and the test dataset. We then ran the LDA on these documents to get a posterior topic distribution over 50 topics. This topic distribution was then used as 50 features in our model.
\end{enumerate}


%!TEX root = cl2-project.tex
\subsection{Seeded LDA}
\label{sec:seededlda}
When we ran LDA on the Facebook statuses, we observed that words that can represent a similar disposition are spread across multiple topics.

Observing the top topic terms output by LDA, we observe that there are some words that occur across topics causing the topic distribution values to get distributed across multiple topics. This way, no one topic gets a high score and this is not helpful in determining what topics are predictive of \textit{depression}. These topics demonstrate similar feeling and if the topic distribution for statuses get distributed across these two topics, we would not get a clear indication of what the underlying feeling of the user was when this was posted. For example, the words \textit{sad}, \textit{laugh}, \textit{feel}, \textit{love} occur in more than one topic in LDA. Another point to notice is that all the topics in Table \ref{table:ldawords_1} signify mixed feelings and do not help in identifying a particular emotion. As depression is usually indicated by use of words that signify \textit{sadness} and lack of depression by words that signify \textit{happiness}, it will be helpful to have topics that capture these emotions separately. Note that smile and sad occur in the same topic, so we cannot distinguish between these using the topic distribution. So, to mitigate this we try to use \textit{SeededLDA}, to guide topic models to learn topics that are of specific interest to us.

Table \ref{table:seedwords_1} refers to the seeds used for \textit{SeededLDA}. On examining some statuses and the top topic terms produced by LDA, we find that, words such as love, like, happy, playful, etc denote happiness and hence are not likely to correlate with depression. On the other hand, words such as hate, cry, disappointed, sad, unhappy, etc are likely to correlate with depression. Also words that represent work and related tension, such as stress, tired, time, busy, can correlate with depression symptoms. So, we seeded the topics with words that we think correlate with depression.

\begin{table*} [ht!]
%\begin{center}
	\begin{tabular}{ l }
\hline
{topic~1: smile, sad, laugh, wink, playful, love, day, crying, surprise, god, feel, movie, book, mood}\\
{topic~2: laugh, today, sad, feel, tired, people, sick, playful, school, miss, sleep, wait, stupid, hope, fun, kind, friend, bad}\\
{topic~3: laugh, life, day, good, great, world, live, fall, hope, mind, end, days, peace, living, night, fear, change, full, sun}\\
{topic~4: laugh, smile, run, love, vain, joke, ruin, night, party}\\
\hline
    \end{tabular}
      \caption{\noindent Seed words for \textit{SeededLDA}}
        \label{table:ldawords_1}
%\end{center}
\end{table*}

We observed that LDA on the Facebook statuses
\begin{table*} [ht!]
%\begin{center}
	\begin{tabular}{ l }
	\hline
{topic~1: work, bad, tired, stress, time, cry, pain}\\
{topic~2: happy, glad, excited, fun, energy, pleasant, love, awesome}\\
{topic~3: love, life, heart, surprise, joyful, smile}\\
{topic~4: unpleasant, unhappy, sad, irritated, hate, jealous, gloomy, disappointed }\\
{topic~5: hate, awful, fuck, indecision, bored}\\
\hline
    \end{tabular}
      \caption{\noindent Seed words for \textit{SeededLDA}}
        \label{table:seedwords_1}
%\end{center}
\end{table*}

\textit{SeededLDA} was run with for \textit{10} topics, of which \textit{5} topics are seeded with words in Table \textit{seedwords\_1} and \textit{5} topics are un-seeded. It was run for \textit{500 iterations} with standard values for $\alpha$ and $\beta$. The topics that are not seeded capture any words in the data that were missed in our seed words. \textit{SeededLDA} finds the words that are related to the seed words and places those words in the same topic. This helps because we don't need to identify all the words corresponding to depression and lack of depression. SeededLDA will gather these words as placing the related words in the same topic.


\begin{table*} [ht!]
%\begin{center}
	\begin{tabular}{ l }
	\hline
{topic~1: work, bad, tired, stress, time, cry, pain}\\
{topic~2: happy, glad, excited, fun, energy, pleasant, love, awesome}\\
{topic~3: love, life, heart, surprise, joyful, smile}\\
{topic~4: unpleasant, unhappy, sad, irritated, hate, jealous, gloomy, disappointed }\\
{topic~5: hate, awful, fuck, indecision, bored}\\
\hline
    \end{tabular}
      \caption{\noindent Top topic terms for \textit{seeded} topics output by \textit{SeededLDA}}
        \label{table:seedwords_1}
%\end{center}
\end{table*}

\subsubsection{Seeded LDA with DSM Features}

We further experimented with seed words from DSM-IV manual. On page ... of the manual, there is instructions for determining if a person is suffering from a risk of depression. These instructions specify symptoms that are observed in depressed patients. An excerpt of the manual is given in Table \ref{} for the sake of convenience. 




\subsection{SHLDA}
\subsection{Word2Vec}
\label{model}

%!TEX root = cl2-project.tex
\subsection{SHLDA}

%!TEX root = cl2-project.tex
\subsection{Word2Vec}

Most current NLP systems treat words as atomic units, i.e., words are represented as a vector of size $V$ (where $V$ is the size of the vocabulary), with the value 1 only at the index of that word and 0 at all other places. The problem with this representation is that there is no notion of similarity between words.

Recently, with the advent of deep learning models which can trained on huge datasets, there has been a lot of interest in the field of learning distributed word representations. The idea in distributed representations of words is to represent words as points in some space (usually a vector space) with the expectation that similar words will be close to each other in that vector space. Words can have multiple degrees of similarity, both syntactically and semantically. Using multidimensional vectors, it is possible to capture all these different degrees of similarity. Another advantage of using these representations in various models is, when model parameters are adjusted in response to a particular word or word sequence, the improvements will carry over to occurrences of similar words and sequences.

These models are able to capture rich semantic information and have been shown to significantly improve NLP applications such as sentiment analysis, POS tagging, NER, etc. In this work, we explored a specific model for learning distributed representations of words called the Skip Gram model. The software package using this model is popularly known as $word2vec$.

The Skip Gram model is a very simple model which can learn distributed representations of words from huge datasets. This model has gained a lot of popularity because of the syntactic and semantic relationships it captures, in the form of linear regularities. More specifically, one can show that the following relationships among the word vectors hold. \\
$vector('car') - vector('cars') \approx vector('chair') - vector('chairs')$ \\
$vector('king') - vector('man') \approx vector('queen') - vector('woman')$

In this work, we attempted to build vector representations of phrases, in this case, status updates for each user. The status updates for each user were concatenated together, and we learnt a vector representation for the text associated to each user. Thus, each user was associated with a vector representation belonging in the same vector space as the words.

We did this by using a very simple technique. Let's suppose we have learnt the individual word representations from some big dataset. We use the preprocessing steps mentioned previously on the statuses for each user. This removes all the stop words and leaves us with statuses contained only of relevant words. Then, for each user, we just do a simple average over all the vectors of the words in each status for that user. In spite of being such a simple technique, this does make intuitive sense. Users who are posting depressed status updates more often would get an average vector more close to the depressed region in the space. On the other hand, people who post positive status updates should lie in a more happier part of the vector space. Using this intuition, we used as features just the average vector representation associated with each user and ran a supervised classification task on it.

We have not yet talked about how we can train the individual word representations in our model. According to us, there are three main approaches one can take towards this while using the $word2vec$ tool. Pre-trained 300 dimensional word vectors are available online which can be used to find the vector representation for each word. These word representations were trained on a Google News corpus containing around a 100 billion words with a vocabulary size of 3 million. However, Google News corpus data is very different from the conversational style of Facebook status updates. Thus, just using these word representations won't give us a very good result. We demonstrate this in the next section, where we show poor classification results on just using these pre-trained vectors. The second approach one can use is to train the Skip Gram model on the unlabeled data set of Facebook status updates that we have been provided. However, this is not feasible because of the following reason. The Skip Gram model is a very simple 2 layer model. While this simplicity in the model helps make the model extremely fast compared to other word representation learning models, it also means that it needs a lot more data to train effectively. Data sets containing at least hundreds of millions of words are required to train the Skip Gram properly. This makes the unlabeled data set of the Facebook status updates unsuitable for training the model.

The final approach one can take, and the one we think has the most promise, is to initialize the Skip Gram model with the pre-trained word vector representations and then to continue training the model using the unlabeled data set of Facebook status updates. This should initialize the word vectors to a good position, and the appropriate domain changes should be captured while we continue training on the unlabeled status updates. The $word2vec$ package available on Google has no way of doing this. The gensim toolkit for word2vec has the capability of doing an online training as mentioned. However, it cannot do so with the pre-trained word vectors learnt from the Google News corpus since it lacks vital information like the vocabulary tree which is required for the gensim code. Thus, we would have to first train our model on the Google News corpus and then do the domain adaptation by continuing training on the unlabeled status updates. This requires large computing power and thus we were unable to pursue this idea.


%!TEX root = cl2-project.tex

\section{Empirical Evaluation}
\label{sec:experiments}

--- Explanation of Models, what went into them\\
--- Tools- weka \\
--- parameters in weka --- naive bayes, crossvalidation 10-fold, smote, \\
--- filtered features, why and how. a table that gives most correlated features ---- infoGain attribute evaluation on weka\\
--- top questions, and features correlated with questions\\
--- graphs for correlated features
\subsection{Depression Prediction Models}


\subsection{Feature Selection}

Out of all the features that we extracted, we should be able to figure out good features from the set of features that help us to predict the set of depressed users better. The method that we adopt here is to choose the set of questions from 20 questions which are indicative of the depressed state of the user. For example, if a particular question has been selected to be indicative of depression, then a depressed user is more likely to give a score of 3 to this question. We describe the procedure of selecting these questions later. Thereafter, we compute the Pearson correlation coefficient ($\rho$) between a feature vector (FV) and the question-answer (QA) vector along with the corresponding $p$-value. A QA vector contains the answers for a given question for all the users. If there are $m$ feature vectors and $n$ QA vectors, then we get an $m\times n$ matrix, where each cell of the matrix is a $(\rho, p)$ tuple. We keep those tuples in the matrix which are above a certain threshold. This ensures that we have pairs of vectors having given $\rho$ value with high confidence. Next, we find the average of the remaining $\rho$ values for a FV to get $\rho_{avg}$. Finally, we sort the feature vectors in a non-increasing order by their $\rho_{avg}$, which gives us the most relevant feature vectors to work with.

The question selection process is as follows: Every question has 4 options (denoted by A0, A1, A2 and A3) and can be answered with a value from 0-3, with the higher value indicative of depressed state of a user. For each question, we aggregate the counts for each possible answer. For example, say we consider answers of 10 users for Question 1; and we find that A0 has count 2, A1 has count 1, A2 has count 3 and A4 has count 4. Since we are trying to identify questions which are indicative of the depressed state of a user, we add up the counts for A2 and A3, and sort the questions in a non-increasing order. We take the top-5 questions for top-25 depressed users and then repeat this procedure for top-50 users, top-100 and top-150 users. Finally, we choose the questions based on the number of times it occurred in the four sets. Essentially, the idea here is to choose the questions which has consistently maintained the top spot in the ordering.




\begin{table} [ht]
\begin{tabular}{lcccc}
\toprule
Models & Precision & Recall & \textit{F1-crossvalidation} & \textit{F1-test}  \\
\midrule
Baseline (B) & 0.7393 & 0.5462 & 0.6673 & 0.5815 \\
B + LDA & \textbf{0.7492} & \textbf{0.5748} & \textbf{0.6923} & \textbf{0.6033} \\
\textit{B + SeededLDA} & \textbf{0.7492} & \textbf{0.5748} & \textbf{0.6923} & \textbf{0.6033} \\
\textit{B + $SeededLDA_{DSM}$} & \textbf{0.7492} & \textbf{0.5748} & \textbf{0.6923} & \textbf{0.6033} \\
\textit{Filtered Feature Set} & \textbf{0.7492} & \textbf{0.5748} & \textbf{0.6923} & \textbf{0.6033} \\
\textit{Regression + Threshold} & \textbf{0.7492} & \textbf{0.5748} & \textbf{0.6923} & \textbf{0.6033} \\
\textit{SLDA } & \textbf{0.7492} & \textbf{0.5748} & \textbf{0.6923} & \textbf{0.6033} \\
\bottomrule
\end{tabular}
\caption{Performance}
\label{table:results_1}
\end{table}


\begin{table} [ht]
\begin{tabular}{lcccc}
\toprule
Models & Precision & Recall & \textit{F1-crossvalidation} & \textit{F1-test}  \\
\midrule
\textit{Filter-5} & \textbf{0.7492} & \textbf{0.5748} & \textbf{0.6923} & \textbf{0.6033} \\
\textit{Filter-10} & \textbf{0.7492} & \textbf{0.5748} & \textbf{0.6923} & \textbf{0.6033} \\
\bottomrule
\end{tabular}
\caption{Performance of question based filtering models}
\label{table:results_2}
\end{table}

\begin{table} [ht]
\begin{tabular}{lcccc}
\toprule
Models & Precision & Recall & \textit{F1-crossvalidation} & \textit{F1-test}  \\
\midrule
\textit{Regression Threshold = 33 } & \textbf{0.7492} & \textbf{0.5748} & \textbf{0.6923} & \textbf{0.6033} \\
\textit{Regression Threshold = 28 } & \textbf{0.7492} & \textbf{0.5748} & \textbf{0.6923} & \textbf{0.6033} \\
\textit{Regression Threshold = 26 } & \textbf{0.7492} & \textbf{0.5748} & \textbf{0.6923} & \textbf{0.6033} \\
\bottomrule
\end{tabular}
\caption{Performance of regression models}
\label{table:results_2}
\end{table}







%!TEX root = cl2-project.tex
\section{Discussion}
\label{sec:disc}
\bibliographystyle{plain}
\bibliography{references}
%\section{Introduction}
%
%\section{Motivation}
%
%\section{PSL Models for Learner Engagement in MOOCs}
%
%\section{Enhancing PSL models using LDA}
%
%\subsection{Traditional LDA}
%\subsection{Coarse-grain seeded LDA}
%\subsection{Fine-grain seeded LDA}
%
%\section{Experimental Setup}
%\section{Results and Discussion}



\end{document}
